\documentclass[10pt]{elsarticle}
%\documentclass[3p,times]{elsarticle}
%\documentclass[review]{elsarticle}
\usepackage{booktabs}
\usepackage[scale=2]{ccicons}
\usepackage{pgfplots}
\usepackage{xspace}
\usepackage{latexsym,amsmath,amssymb,amsfonts,epsfig}
\usepackage{ragged2e}
\usepackage[nooneline]{subfigure}
\usepackage[T1]{fontenc}
\usepackage{lmodern}
\usepackage{tabularx}
\usepackage[flushleft]{threeparttable}
\usepackage{eqnarray,amsmath}
\usepackage{stackrel}
\usepackage{algorithm}
\usepackage[nooneline]{subfigure}
\usepackage[T1]{fontenc}
\usepackage{lmodern}
\usepackage{tabularx}
\usepackage[flushleft]{threeparttable}
\usepackage{url}
\usepackage{graphicx}
\usepackage{adjustbox}
\usepackage{epstopdf}
\usepackage{amsmath}
\usepackage{amsthm}
\usepackage[english]{babel}
\usepackage[applemac]{inputenc}
\usepackage{mathptmx}
\usepackage[flushleft]{threeparttable}
\usepackage[font=small,skip=3pt]{caption}
\usepackage{url}
\usepackage{xcolor}
\usepackage{caption}
\usepackage{algorithmic,algorithm}
\usepackage{multirow,multicol}
\usepackage{tikz}
\usetikzlibrary{datavisualization}
\usetikzlibrary{datavisualization.formats.functions}
\usetikzlibrary{calc}
\usepackage{pgf-umlsd}
\usepackage{moreverb}
\usepackage{fix-cm}
\usepackage{hyperref}
\usepackage{soul}
\newcommand{\hll}[1]{\colorbox{yellow}{$\displaystyle #1$}}

\makeatletter
\usepackage[margin=1 in]{geometry}
\newcommand\xleftrightarrow[2][]{%
\ext@arrow 9999{\longleftrightarrowfill@}{#1}{#2}}
\newcommand\longleftrightarrowfill@{%
\arrowfill@\leftarrow\relbar\rightarrow}
\makeatother

%%%%%%%%%%%%%%%%%%%%%%%%%%%%%%%%%%%%%%%%%%%%%%%%%%%%%%%%%%%%%%%
%%%%%%%%%%%%%%%%%%%%%%%%%%%%%%%%%%%%%%%%%%%%%%%%%%%%%%%%%%%%%%%
\newtheorem{theorem}{Theorem}
\newtheorem{lemma}{Lemma}
\newtheorem{mydef}{Definition}
\newtheorem{rem}{Remarks}
%%%%%%%%%%%%%%%%%%%%%%%%%%%%%%%%%%%%%%%%%%%%%%%%%%%%%%%%%%%%%%%
%%%%%%%%%%%%%%%%%%%%%%%%%%%%%%%%%%%%%%%%%%%%%%%%%%%%%%%%%%%%%%%


\begin{document}
\begin{frontmatter}
\title{Provably Secure Two-Party Authenticated Key Agreement Protocol for Post-Quantum Environments}

\author{SK Hafizul Islam\fnref{label2},$^*$}
\ead{hafi786@gmail.com} \fntext[label2]{Department of Computer
Science and Engineering, Indian Institute of Information Technology
Kalyani, West Bengal 741235, India.}

\cortext[cor1]{Corresponding author. (SK Hafizul Islam)}


\begin{abstract}
A two-party authenticated key agreement (2PAKA) protocol is a
cryptographic tool employed widely to allow two users to generate a
shared and fresh session key between them in each session over an
insecure network. The authenticated version of a two-party key
agreement protocol is popular because it can easily withstand the
impersonation of the user. In the literature, many 2PAKA protocols
have been put forward with the intractability assumptions of the
discrete logarithm (DLP) problem and integer factorization problem
(IFP). Some recent studies showed that the 2PAKA protocols based on
these assumptions are insecure in post-quantum environments. To
resolve this issue, we have designed a lattice-based 2PAKA
(LB-2PAKA) protocol with the intractability of the
ring-learning-with-errors (RLWE) problem. The proposed LB-2PAKA
protocol is also analyzed in the random oracle model to measure
provable security and to estimate the breaching time. To evaluate
the performance, we used the LatticeCrypto Library and estimated the
running time of our LB-2PAKA protocol. Besides, we analyzed the
communication cost requirement of our LB-2PAKA protocol.
\begin{keyword}
Authenticated Key Agreement, Lattice, Provable security, Random
Oracle Model, Ring-Learning-with-Errors.
\end{keyword}
\end{abstract}
\end{frontmatter}

\section{Introduction}
\label{L1} A two-part key agreement (2PKA) protocol is a
cryptographic primitive that allows two users to set up a common and
fresh session key in each session among them over an insecure
network. In this kind of protocol, two users are involved: (i)
\emph{initiator}, who will initiate the protocol, and (ii)
\emph{responder}, who will respond to the \emph{initiator}. In the
literature, an unauthenticated version of the 2PKA protocol was
introduced by Diffie and Hellman \cite{Diffie1976}. However, this
unauthenticated protocol was vulnerable to the man-in-the-middle
attack. The main reason is that a sender sends a message without
signing it to a receiver, and an adversary can replace the message
by a forged message, which can not be detected by the receiver.
Therefore, the authenticated key agreement (AKA) protocol is popular
for practical applications. In an AKA protocol, each user can verify
other users, so that an adversary cannot impersonate any user and
hence, cannot perform the man-in-the-middle attack.

\subsection{Motivations and Contributions}\label{L11}
We have designed a two-party AKA (2PAKA) protocol in this paper. In
the literature, many 2PAKA protocols
\cite{Cao2010,holbl2012,dang2018,Islamgpwpc2015,Tseng2007} have been
put forward based on the integer factorization problem (IFP), the
RSA problem, the discrete logarithm problem (DLP), or the
computational Diffie-Hellman (CDH) problem. However, some recent
studies \cite{PWShor1994,PWShor1999} showed that these hard problems
are vulnerable by a polynomial time-bounded algorithm in quantum
computing environments. Recently, the NIST has inaugurated a
post-quantum crypto project\cite{NIST2016} to design and standardize
robust cryptographic tools in the post-quantum environments.
Recently, one of the mathematical tools, known as \emph{ideal
lattice} has attracted security scholars to design secure
cryptographic mechanisms based on lattices due to the following
reasons: (i) the operations performed in a lattice are much faster
than other operations executed in a non-lattice-based protocol
\cite{FengHe2018}, (ii) lattice-based cryptographic tools are
designed with the intractability assumption of the worst-case
problems, while the traditional cryptographic protocols are devised
using the average-case problems, and (iii) the hardness assumptions
on the ideal lattices can resist the quantum computing attacks. In
contrast, non-lattice based computational problems are vulnerable in
the same environments. Accordingly, we have motivated to design a
lattice-based 2PAKA (LB-2PAKA) protocol suitable for post-quantum
environments. Our LB-2PAKA protocol exhibit the following merits:
\begin{description}
\item[\textbf{(1)}]~We proposed an LB-2PAKA protocol on ideal lattices based on the ring-learning-with-errors (RLWE) problem. The RLWE problem is assumed as hard as some computational problems in the worst-case on ideal lattices.
\item[\textbf{(2)}]~The proposed LB-2PAKA protocol is analyzed in the random oracle model (ROM), which exhibits that it is provably secure against the RLWE problem on ideal lattices.
\item[\textbf{(3)}]~In our LB-2PAKA protocol, two users can establish a secure session key among them over an insecure network. To avoid the ephemeral secret leakage (ESL) attack, we included the long-term static private keys of the users and the ephemeral secrets selected by the users in the session key.
\item[\textbf{(4)}]~The performance of our LB-2PAKA protocol is evaluated using the LatticeCrypto Library \cite{LCL2019}, which is used popularly in many lattice-based protocols \cite{FengHe2018,Saraswathy2017}.
\item[\textbf{(5)}]~The informal security analysis demonstrated that our LB-2PAKA protocol could provide resilience against the known active and passive attacks.
\end{description}

\subsection{Reviews on Lattice-based Protocols}\label{L12}
In the literature, many cryptographic scholars have initiated to
proposed some lattice-based cryptographic protocols. In 2007, Ding
et al. \cite{DingJ2007} firstly put forwarded an unauthenticated
Diffie-Hellman key agreement protocol using a password. The protocol
is analyzed to be provably secure in the ROM with the intractability
assumption of the RLWE problem on ideal lattices. In the ROM, Ding
et al. \cite{Saraswathy2017} designed two different client-server
authentication (CSA) protocols with the intractability assumption of
the RLWE problem. Later on, Feng et al. \cite{FengHe2018} put
forwarded a provably secure and anonymous CSA protocol for mobile
users with the intractability assumption of the RLWE problem on
ideal lattices. Zhang et al. \cite{ZhangJ2015} firstly proposed a
provably secure and two-pass lattice-based unauthenticated two-party
key agreement protocol with the intractability assumption of the
RLWE problem. In this protocol, the initiator $U_{i}$ draws the
random samples $r_{i}$, $f_{i}$ $\xleftarrow[]{R}\chi_{\beta}$,
computes $x_{i} = \textbf{\emph{a}}\cdot r_{i} +
2\cdot\textbf{\emph{f}}_{i}$ and sends $\{x_{i}\}$ to the responder
$U_{j}$. After receiving the message $\{x_{i}\}$, the responder
$U_{j}$ draws the random samples $r_{j}$, $f_{j}$
$\xleftarrow[]{R}\chi_{\beta}$, computes the message $\{y_{j},
w_{j}\}$ and sends it to $U_{i}$. Note that no mechanism (i.e.,
signature \cite{debiao2019}, or message authentication code (MAC)
\cite{islam2016}) has been adapted by $U_{i}$ and $U_{j}$ to verify
the integrity of the messages $\{x_{i}\}$ and $\{y_{j}, w_{j}\}$ if
changed during the communication. Accordingly, the protocol in
\cite{ZhangJ2015} is vulnerable to the man-in-the-middle attack.

\subsection{Organization}
\label{L14}The remaining part of this paper is structured as
follows. Section~\ref{L2} discusses the basics of ideal lattices and
related hardness problems on it. Section~\ref{L3} proposes an
adversarial model for an LB-2PAKA protocol. Section~\ref{L4} explain
our LB-2PAKA protocol. Section~\ref{L5} justifies the correctness
and security analysis of our LB-2PAKA protocol. Section~\ref{L6}
analyze the computation and communication costs of our LB-2PAKA
protocol. Finally, Section~\ref{L7} concludes the paper.

\section{Preliminaries}
\label{L2} Let $(x^{n}+1)\in\mathbb{Z}[x]$ is an irreducible
polynomial or prime polynomial over integers, where $n$ signify the
security parameter, $n = 2^{t}$, for some positive $t\in\mathbb{Z}$.
Let $\mathbb{Z}[x]$ and $\mathbb{Z}_{q}[x]$ signify the rings of
polynomials over $\mathbb{Z}$ and $\mathbb{Z}_{q}$, respectively,
where $q~\mbox{mod}~2n\equiv1$ ($q = 2^{\omega(\mbox{log}_{2}n)}+1$)
be a public prime modulus. Let $\mathbb{R} = \mathbb{Z}[x]/(x^{n} +
1)$ be the ring of integer polynomials modulo $(x^{n}+1)$ over
$\mathbb{Z}$ and $\mathbb{R}_{q} = \mathbb{Z}_{q}[x]/(x^{n} + 1)$ is
the rings of polynomials modulo $(x^{n}+1)$ and each coefficient is
reduced modulo $q$ over $\mathbb{Z}_{q}$. The number of element of
$\mathbb{R}_{q}$ is $q^{n}$ and each element of $\mathbb{R}_{q}$ is
a polynomial of degree less than $n$ whose coefficients belongs to
the set $\mathbb{Z}_{q} = \{0, 1, \cdots, q-1\}$
\cite{LyubashevskyyV2010}.

\begin{lemma}\label{LM1}
\normalfont Given any $\textbf{\emph{a}},
\textbf{\emph{b}}\in\mathbb{R}$, we have
$||\textbf{\emph{a}}\cdot\textbf{\emph{b}}||_{2}\leq\sqrt{n}||\textbf{\emph{a}}||_{2}\cdot||\textbf{\emph{b}}||_{2}$
and $||\textbf{\emph{a}}\cdot\textbf{\emph{b}}||_{\infty}\leq
n||\textbf{\emph{a}}||_{\infty}\cdot||\textbf{\emph{b}}||_{\infty}$
\cite{Saraswathy2017}, where $||\cdot||_{2}$ and
$||\cdot||_{\infty}$ are the $L_{2}$ norm $L_{\infty}$ norm
\cite{FengHe2018}.
\end{lemma}

\begin{lemma}\label{LM2}
\normalfont Given any fixed positive real number $\beta =
\omega(\sqrt{\mbox{log}_{2}n})$, we have
$\mbox{\textbf{Pr}}_{\textbf{\emph{a}}\leftarrow\chi_{\beta}}[||\textbf{\emph{a}}||_{2}>\beta\cdot\sqrt{n}]\leq2^{-n+1}$
\cite{MicciancioDRO2007}, where $\chi_{\beta}$ signify the
\emph{discrete Gaussian distribution} over $\mathbb{R}_{q}$. Let $E
= \{-\left \lfloor{\frac{q}{4}}\right\rfloor, \cdots, \left
\lfloor{\frac{q}{4}}\right\rfloor\}$ be the middle set of
$\mathbb{Z}_{q} = \{-\frac{q}{2}, \cdots, \frac{q}{2}\}$. For any
$x\in\mathbb{Z}_{q}$, the characteristic function \textbf{Cha} is
described as follows:
\[\mbox{Cha}(x) =
  \begin{cases}
    0~~~x\in E\\
    1~~~x\notin E
  \end{cases}
\]
The auxiliary modular function
$\mbox{\textbf{Mod}}_{2}:\mathbb{Z}_{q}\times\{0,1\}\rightarrow\{0,
1\}$ is described as
$\mbox{\textbf{Mod}}_{2}(\textbf{\emph{v}},\textbf{\emph{b}}) =
(\textbf{\emph{v}} +
\textbf{\emph{b}}\cdot\frac{q-1}{2})~\mbox{mod}~q~\mbox{mod}~2$,
where $\textbf{\emph{v}}\in\mathbb{Z}_{q}$ and $\textbf{\emph{b}} =
\mbox{\textbf{Cha}}(\textbf{\emph{v}})$ \cite{ZhangJ2015}.
\end{lemma}

\begin{lemma}\label{LM3}
\normalfont Let $q\geq3$ be a odd prime number and $n = 2^{t}$ is
the security parameter, for some positive $t\in\mathbb{Z}$ and $q =
2^{\omega(\mbox{log}_{2}n)}+1$. Let
$\textbf{\emph{v}}\in\mathbb{Z}_{q}$ be selected uniformly at
random. For any $b\in\{0, 1\}$ and any
$\textbf{\emph{w}}\in\mathbb{Z}_{q}$, the output distribution of
$\mbox{\textbf{Mod}}_{2}(\textbf{\emph{v}}+\textbf{\emph{w}}, b)$
given $\mbox{\textbf{Cha}}(\textbf{\emph{v}}))$ is statistically
close to uniform on $\{0, 1\}$ \cite{ZhangJ2015}.
\end{lemma}

\begin{lemma}\label{LM4}
\normalfont Given a prime number $q\geq3$ and $\textbf{\emph{a}},
\textbf{\emph{e}}\in\mathbb{R}_{q}$ such that $|e|<\frac{q}{8}$, we
have $\mbox{\textbf{Mod}}_{2}(\textbf{\emph{a}},
\mbox{\textbf{Cha}}(\textbf{\emph{a}})) =
\mbox{\textbf{Mod}}_{2}(\textbf{\emph{w}},
\mbox{\textbf{Cha}}(\textbf{\emph{a}}))$, where $\textbf{\emph{w}} =
\textbf{\emph{a}} + 2\cdot\textbf{\emph{e}}$ \cite{ZhangJ2015}.
\end{lemma}

\noindent The functions \textbf{Cha} and $\mbox{\textbf{Mod}}_{2}$
can be extended on $\mathbb{R}_{q}$ as follows: Given
$\textbf{\emph{a}} = a_{0} + a_{1}x + a_{2}x^{2} + \cdots +
a_{n-1}x^{n-1}\in\mathbb{R}$, we considered it as a vector
$\textbf{\emph{a}}$ $=$ $(a_{0}$, $a_{1}$, $\cdots$, $a_{n-1})$. For
$\textbf{\emph{v}}$ $=$ $(v_{0}$, $v_{1}$, $\cdots$,
$v_{n-1})\in\{0,1\}^{n}$, we have
$\mbox{\textbf{Cha}}(\textbf{\emph{a}})$ $=$
$(\mbox{\textbf{Cha}}(a_{0})$, $\mbox{\textbf{Cha}}(a_{1})$,
$\cdots$, $\mbox{\textbf{Cha}}(a_{n-1}))$ and
$\mbox{\textbf{Mod}}_{2}(\textbf{\emph{a}}$, $\textbf{\emph{v}})$
$=$ $(\mbox{\textbf{Mod}}_{2}(a_{0}$, $v_{0})$,
$\mbox{\textbf{Mod}}_{2}(a_{1}$, $v_{1})$, $\cdots$,
$\mbox{\textbf{Mod}}_{2}(a_{n-1}$, $v_{n-1}))$ \cite{ZhangJ2015}.

\begin{mydef}[Ring Learning with Errors (RLWE)]
\normalfont Let $\textbf{\emph{s}}\in\mathbb{R}_{q}$ and
$\textbf{\emph{A}}_{\textbf{\emph{s}}, \chi_{\beta}}$ be the
distribution over $(\textbf{\emph{a}},
\textbf{\emph{a}}\cdot\textbf{\emph{s}} +
2\cdot\textbf{\emph{e}})\in\mathbb{R}_{q}\times\mathbb{R}_{q}$,
where $\textbf{\emph{a}}$ is randomly selected from $\mathbb{R}_{q}$
and $\textbf{\emph{e}}\xleftarrow[]{R}\chi_{\beta}$. It is
computationally infeasible for a probabilistic polynomial
time-bounded (PPT) algorithm $\mathcal{B}$ to distinguish the
distribution $\textbf{\emph{A}}_{\textbf{\emph{s}}, \chi_{\beta}}$
from the uniform distribution on
$\mathbb{R}_{q}\times\mathbb{R}_{q}$, for given a fixed
$\textbf{\emph{a}}$ sampled from $\chi_{\beta}$ and polynomially
many samples \cite{LyubashevskyyV2010}.
\end{mydef}

\begin{mydef}[Pairing with errors (PWE)]
\normalfont Let $\textbf{\emph{x}},
\textbf{\emph{s}}\in\mathbb{R}_{q}$, we describe
$\psi(\textbf{\emph{x}}, \textbf{\emph{s}}) =
\mbox{\textbf{Mod}}_{2}(\textbf{\emph{x}}\cdot\textbf{\emph{s}},
\mbox{\textbf{Cha}}(\textbf{\emph{x}}\cdot\textbf{\emph{s}}))$. For
given $\textbf{\emph{a}}, \textbf{\emph{x}},
\textbf{\emph{y}}\in\mathbb{R}_{q}$, it is computationally
infeasible for any $\mathcal{B}$ to compute $\psi(\textbf{\emph{x}},
\textbf{\emph{s}})$, where $\textbf{\emph{y}} =
\textbf{\emph{a}}\cdot\textbf{\emph{s}} + 2\cdot\textbf{\emph{e}}$
for the unknowns $\textbf{\emph{s}},
\textbf{\emph{e}}\xleftarrow[]{R}\chi_{\beta}$.
\end{mydef}


\begin{mydef}[Decision pairing with errors (DPWE)]
\normalfont Given $\textbf{\emph{a}}$, $\textbf{\emph{x}}$,
$\textbf{\emph{y}}\in\mathbb{R}_{q}$ and $\omega,
\sigma\in\{0,1\}^{*}$, where $\omega =
\mbox{\textbf{Cha}}(\textbf{\emph{k}})$ for
$\textbf{\emph{x}}\in\mathbb{R}_{q}$ and $\sigma =
\mbox{\textbf{Mod}}_{2}(\textbf{\emph{k}}, \omega)$. It is
computationally infeasible for any $\mathcal{B}$ to determine
whether $\textbf{\emph{k}} = \textbf{\emph{x}}\cdot\textbf{\emph{s}}
+ 2\cdot\textbf{\emph{g}}$ and $\textbf{\emph{y}} =
\textbf{\emph{a}}\cdot\textbf{\emph{s}} + 2\cdot\textbf{\emph{e}}$
for the unknowns $\textbf{\emph{s}}, \textbf{\emph{g}},
\textbf{\emph{e}}\in \chi_{\beta}$ or $(\textbf{\emph{k}},
\textbf{\emph{y}})$ is uniformly random in
$\mathbb{R}_{q}\times\mathbb{R}_{q}$.
\end{mydef}

\section{Adversary Model}
\label{L3}Here, we explain the formal adversarial model of an
LB-2PAKA protocol based on the computational model called the random
oracle model (ROM), which was introduced in \cite{Bellare1993}. This
model tests the semantic security of the session key and the mutual
authentication between two users. In our adversarial model, we
assume that $\mathbb{U} = \{U_{1}, U_{2}, \cdots, U_{n}\}$ be a set
of $n$ users and any two distinct users from $\mathbb{U}$ can
cooperatively execute the proposed LB-2PAKA protocol. The set
$\mathbb{U}$ includes both honest users and dishonest users. The
dishonest user may be a malicious insider user or an outsider user.
We assume that each $U_{i}$ ($1\leq i\leq n$) must have their
long-term static private and public keys $\langle(s_{i}$, $e_{i}$),
$p_{i}\rangle$ and issued the public key certificate for the public
key $p_{i}$ from a trusted certificate authority (CA). The proposed
LB- 2PAKA protocol can be viewed as a collection of $n$ programs,
which are cooperatively executed by different $n$ pairs of users. An
instance of the protocol among two different users is considered as
a session, and each user may execute multiple sessions
simultaneously with different users. The communication in every
session is assumed to be fully monitored by an adversary
$\mathcal{A}$, who can modify, insert, or delete the messages
exchanges between two users.

We denote the oracle $\Pi^{k}_{U_i,U_j}$ is the $k$th instance of
the user $U_{i}$ involved with its partner $U_{j}$ in a session $k$.
The session identifier $sid^{k}_{U_i,U_j}$ is considered as the
concatenation of the messages exchanged among $U_{i}$ with identity
$id_{i}$ and $U_{j}$ with identity $id_{j}$. The partner identifier
$pid^{k}_{U_i,U_j}$ of $\Pi^{k}_{U_i,U_j}$ is a set containing the
identity $id_{i}$ of $U_{i}$ and the identity $id_{j}$ of the party
$U_{j}$ with whom $U_{i}$ wants to establish a session. The semantic
security of the session key and mutual authentication between the
involved users in a session of LB-2PAKA protocol is defined by a
series of challenge-response games, which were executed by a
challenger $\mathcal{C}$ and an adversary $\mathcal{A}$. In each
game, $\mathcal{A}$ must solve a challenge on a \emph{\textbf{Test}}
session to violate the semantic security and mutual authentication
of LB-2PAKA protocol. In each game, $\mathcal{A}$ has given the
capability to execute the following queries to $\mathcal{C}$, which
in turn, returns the answer of the corresponding queries to
$\mathcal{A}$:

\begin{itemize}
\item\textbf{Execute($\Pi^{k}_{U_i,U_j}$):}~$\mathcal{A}$
can ask an \textbf{Execute($\Pi^{k}_{U_i,U_j}$)} query to
$\mathcal{C}$ to get the messages computed as per the description of
the protocol among the instances $\Pi^{k}_{U_i}$ and
$\Pi^{k}_{U_j}$.

\item \textbf{Send($\Pi^{k}_{P}$, $m$):}~$\mathcal{A}$ can ask a
\textbf{Send($\Pi^{k}_{P}$, $m$)} query to acquire the messages
spawned by the instance $\Pi^{k}_{P}$. Note that $\mathcal{A}$ can
also commence the protocol by executing a
\textbf{\mbox{Send}($\Pi^{k}_{P}$, \emph{start})} query to
$\mathcal{C}$.

\item \textbf{Corrupt($P$):}~This query models an insider attack
and an unknown key share attack. To compromise an entity
$P\in\mathbb{U}$ and to acquire the long-term private key of $P$,
$\mathcal{A}$ can ask a \textbf{Corrupt($P$)} query to $\mathcal{C}$
and then $\mathcal{C}$ will return the long-term private key of $P$
to $\mathcal{A}$.

\item \textbf{Reveal($\Pi^{k}_{U_{i},U_{j}}$):}~The execution of this query
models an active attack, called known-key attack. With a
\textbf{Reveal($\Pi^{k}_{U_{i},U_{j}}$)}, $\mathcal{A}$ obtains the
session key $SK$ for misusing it. Assume that $SK$ is freshly
computed in the $k$th session among the instances $\Pi^{k}_{U_{i}}$
and $\Pi^{k}_{U_{j}}$, and $\Pi^{k}_{U_{i}}$ and it's partner
$\Pi^{k}_{U_{j}}$ enter in to the \emph{accepted} state in the
session $k$. Suppose $\mathcal{A}$ sends a
\textbf{Reveal($\Pi^{k}_{U_{i},U_{j}}$)} query to $\mathcal{C}$,
then $\mathcal{C}$ returns the answers as follows:

\begin{itemize}
\item $\mathcal{C}$ returns the session key $SK$ to $\mathcal{A}$,
provided $\Pi^{k}_{U_i}$ and it's partner $\Pi^{k}_{U_j}$ are in
\emph{accepted} state and they freshly compute a correct session key
$SK$.

\item Otherwise, $\mathcal{C}$ returns \emph{null} value to $\mathcal{A}$ as output.
\end{itemize}

\item \textbf{RevealEph($\Pi^{k}_{U_i,U_j}$):}~To obtain the
ephemeral secrets used by the participants $U_i$ and $U_j$ in a
session $k$, $\mathcal{A}$ can ask a
\textbf{RevealEph($\Pi^{k}_{U_i,U_j}$)} query to $\mathcal{C}$, then
$\mathcal{C}$ returns the ephemeral secrets used in the session $k$.
This query models the goal of the ephemeral secret leakage attack.

\item \textbf{Test($\Pi^{k}_{U_{i},U_{j}}$):}~This query evaluates
the semantic security of $SK$. $\mathcal{A}$ is allowed to send only
a single \textbf{Test($\Pi^{k}_{U_{i},U_{j}}$)} query to
$\mathcal{C}$ for each session $k$. In response to a
\textbf{Test($\Pi^{k}_{U_{i},U_{j}}$)} query, $\mathcal{C}$ returns
$SK$ if it is computed between $\Pi^{k}_{U_i}$ and its partner
$\Pi^{k}_{U_j}$, else, returns a \emph{null value}. When
$\mathcal{A}$ simulates a \textbf{Test($\Pi^{k}_{U_{i},U_{j}}$)}
query, $\mathcal{C}$ flips an unbiased coin $b$ and outputs $SK$ as
answer if $b = 1$ holds, else, chooses a \emph{random value} from
$\{0,1\}^{l}$ as answer for the
\textbf{Test($\Pi^{k}_{U_{i},U_{j}}$)} query asked by $\mathcal{A}$.
\end{itemize}

\begin{mydef}[Accepted state]
\label{D7} \normalfont An oracle $\Pi^{k}_{U_i,U_j}$ is called
\emph{accepted} if (i) the instances $\Pi^{k}_{U_i}$ and
$\Pi^{k}_{U_j}$ are mutually authenticate each other in a fresh
session $k$ and holds the same session key $SK$, (ii)
$\Pi^{k}_{U_i}$ and $\Pi^{k}_{U_j}$ hold a session identifier
$sid^{k}_{U_i,U_j}$, and (3) $\Pi^{k}_{U_i}$ and $\Pi^{k}_{U_j}$
hold a partner identifier $pid^{k}_{U_i,U_j}$.
\end{mydef}

\begin{mydef}[Partner]
\label{D8} \normalfont Two oracles $\Pi^{k}_{U_i,U_j}$ and
$\Pi^{l}_{U_j,U_i}$ are said to be \emph{partnered} \emph{iff} the
following holds: (1) $\Pi^{k}_{U_i,U_j}$ and $\Pi^{l}_{U_j,U_i}$
entered into \emph{accepted} state, (ii) $sid^{k}_{U_i,U_j} =
sid^{l}_{U_j,U_i}$, and (ii) $pid^{k}_{U_i,U_j} =
pid^{l}_{U_j,U_i}$.
\end{mydef}

\begin{mydef}[Freshness of a session]
\label{D9} \normalfont An instance $\Pi^{k}_{U_i,U_j}$ is
\emph{fresh} if the following holds: (i) $\Pi^{k}_{U_i,U_j}$ is in
the \emph{accepted} state, (ii) No
\textbf{Reveal($\Pi^{k}_{U_i,U_j}$)} query should be executed, (iii)
No \textbf{Send($\Pi^{k}_{U_i}$, $m$)} or
\textbf{Send($\Pi^{k}_{U_j}$, $m$)} query should be executed, (iv)
Only \textbf{Corrupt}($U_i$) or \textbf{Corrupt}($U_j$) can be
executed.
\end{mydef}


\begin{mydef}[Session key secrecy]
\label{D9} \normalfont An LB-2AKA protocol satisfy the session key
secrecy, if no polynomial time bounded adversary $\mathcal{A}$ has a
non-negligible advantage in the following game played between
$\mathcal{A}$ and infinite set of oracles $\Pi^{k}_{U_i,U_j}$: (i) A
long-term private key is assigned to each user $U_{i}\in\mathbb{U}$,
(ii) $\mathcal{A}$ may ask several queries and get back the response
from the corresponding oracles, (iii) There is no
\textbf{Reveal($\Pi^{k}_{U_i,U_j}$)} query or
\textbf{Corrupt($U_{i}$)}/\textbf{Corrupt($U_{j}$)} query being
asked before the \textbf{Test($\Pi^{k}_{U_i,U_j}$)} query has been
asked, and (iv) $\mathcal{A}$ may ask other queries during asking
the \textbf{Test($\Pi^{k}_{U_i,U_j}$)} query where
$\Pi^{k}_{U_i,U_j}$ is fresh. $\mathcal{A}$ outputs its guess bit
$b^{\prime}$ for the bit $b$, which is chosen in the
\textbf{Test($\Pi^{k}_{U_i,U_j}$)} query eventually and the game is
successfully terminated.
\end{mydef}

\begin{mydef}[Probability of success]
\label{D10} \normalfont Let $Succ(\mathcal{A})$ be the event that
$\mathcal{A}$ makes a single  \textbf{Test($\Pi^{k}_{U_i,U_j}$)}
query to some fresh session $k$ to the instance $\Pi^{k}_{U_i,U_j}$
that has end successfully, and eventually returns a guess bit
$b^{\prime}$, where $b^{\prime} = b$, the bit $b$ was selected in
the \emph{\textbf{Test}} query. The advantage, i.e., probability of
success of $\mathcal{A}$ with in the polynomial time bound $t$ in
violating the semantic security of the session key of our LB-2PAKA
protocol is defined as $Adv^{LB-2PAKA}_{\mathcal{A}}(t) =
|Pr[Succ(\mathcal{A})]- \frac{1}{2}|$.
\end{mydef}

\begin{mydef}[Semantic security]
\label{D11} \normalfont The proposed LB-2PAKA protocol is
semantically secure if (i) in the presence of $\mathcal{A}$,
$\Pi^{k}_{U_i}$ and its partner $\Pi^{k}_{U_j}$ are in
\emph{accepted} state, and (ii) $Adv^{LB-2PAKA}_{\mathcal{A}}(t)$ is
negligible.
\end{mydef}

%%%%%%%%%%%%%%%%%%%%%%%%%%%%%%%%%%%%%%%%%%%%%%%%%%%%%%%%%%%%%%%%%%%%%%%%%%%%%
%%%%%%%%%%%%%%%%%%%%%%%%%%%%%%%%%%%%%%%%%%%%%%%%%%%%%%%%%%%%%%%%%%%%%%%%%%%%%
\begin{table}[h]
\centering \caption{Notations} \label{tab1} \scalebox{1}{
\begin{tabular}{|l|l|}
\hline
\textbf{Notation}    & \textbf{Meaning}\\
\hline
$q$                  & Odd prime number\\
\hline
$\chi_{\beta}$       & Discrete Gaussian distribution for $\beta>0$\\
\hline
$\textbf{\emph{x}}\xleftarrow[]{R}\chi_{\beta}$ & $\textbf{\emph{x}}$ is sampled from $\mathbb{R}_{q}$ uniformly at random according to $\chi_{\beta}$\\
\hline
$\mathbb{R}$, $\mathbb{R}_{q}$   & Ring, where $\mathbb{R} = \mathbb{Z}[x]/(x^{n} + 1)$ and $\mathbb{R}_{q} = \mathbb{Z}_{q}[x]/(x^{n} + 1)$\\
\hline
SA                   & System administrator\\
\hline
CA                   & Certificate authority\\
\hline
$U_{i}$/$U_{j}$      & Initiator (User)/Responder (User)\\
\hline
$id_{i}$             & Identity of $U_{i}$\\
\hline
($s_{i}$, $e_{i}$)  & long-term static private key of $U_{i}$\\
\hline
$p_{i}$             & Long-term static public key of $U_{i}$, $p_{i} = \textbf{\emph{a}}\cdot s_{i} + 2\cdot\textbf{\emph{e}}_{i}$\\
\hline
$H_{j}(\cdot)$      & Cryptographic hash functions, $H_{j}:\{0, 1\}^{*}\rightarrow\{0,1\}^{l}$, $j = 1,2,3$\\
\hline
\end{tabular}}
\end{table}
%%%%%%%%%%%%%%%%%%%%%%%%%%%%%%%%%%%%%%%%%%%%%%%%%%%%%%%%%%%%%%%%%%%%%%%%%%%%%
%%%%%%%%%%%%%%%%%%%%%%%%%%%%%%%%%%%%%%%%%%%%%%%%%%%%%%%%%%%%%%%%%%%%%%%%%%%%%

\section{Proposed LB-2PAKA Protocol}
\label{L4}Here, we propose a new LB-2PAKA protocol in which three
entities are involved: i) the initiator $U_{i}$, ii) the responder
$U_{j}$, and iii) the system administrator (SA). The role of SA is
to generate all the system parameters. The systems parameters are
essential to perform the execution of�the proposed protocol.
Table~\ref{tab1} lists the different notations and their meanings.
In our LB-2PAKA protocol, both the users $U_{i}$ and $U_{j}$
generate their pair of keys: a static long-term private key and a
corresponding static long-term public key. The static long-term
public keys of $U_{i}$ and $U_{j}$ must be authenticated by a
certificate authority (CA) before using them. We describe our
LB-2PAKA protocol with the following phases.

\subsection{Initialization Phase}\label{L41}
This phase is executed by the SA to generate the system parameters.
SA selects a sufficiently large odd prime
$q>16\beta^{2}n^{\frac{3}{2}}$ such that $q~\mbox{mod}~2n\equiv1$,
where $n\in\mathbb{Z}_{q}$, $n = 2^{t}$ for some positive
$t\in\mathbb{Z}$. SA further selects $\mathbb{R} =
\mathbb{Z}[x]/(x^{n} + 1)$, $\mathbb{R}_{q} =
\mathbb{Z}_{q}[x]/(x^{n} + 1)$, a discrete Gaussian distribution
$\chi_{\beta}$, a fixed element $\textbf{\emph{a}}$ uniformly at
random from $\mathbb{R}_{q}$ and  three one-way secure hash
functions $H_{1}$, $H_{2}$,
$H_{3}:\{0,1\}^{*}\rightarrow\{0,1\}^{l}$. In our protocol, we
choose the SHA-512 hash function for hashing. SA publishes $\langle
n$, $q$, $\chi_{\beta}$, $H_{1}(\cdot)$, $H_{2}(\cdot)$,
$H_{3}(\cdot)\rangle$ as system parameters.

\subsection{Key Generation Phase}\label{L42}
User $U_{i}$ with an identity $id_{i}$ generates his/her long-term
static secret key as $(s_{i}$, $e_{i})$
$\xleftarrow[]{R}\chi_{\beta}$ and computes his/her static long-term
public key as $p_{i} = \textbf{\emph{a}}\cdot s_{i} +
2\cdot\textbf{\emph{e}}_{i}\in\mathbb{R}_{q}$.

\subsection{Authenticated Key Agreement Phase}\label{L42}
\noindent\textbf{(a)}~The initiator $U_{i}$ draws two samples
$r_{i}$, $f_{i}$ $\xleftarrow[]{R}\chi_{\beta}$ and computes the
followings:
\begin{itemize}
\item[]$x_{i} = \textbf{\emph{a}}\cdot r_{i} + 2\cdot\textbf{\emph{f}}_{i}$
\item[]$t_{i_1} = s_{i}\cdot p_{j}$
\item[]$w_{i_1} = \mbox{\textbf{Cha}}(t_{i_1})$
\item[]$\sigma_{i_1} = \mbox{\textbf{Mod}}_{2}(t_{i_1}, w_{i_1})$
\item[]$\alpha_{i_1} = H_{1}(id_{i}, x_{i}, \sigma_{i_1})$
\end{itemize}
\noindent $U_{i}$ sends the message $\langle id_{i}$, $x_{i}$,
$w_{i_1}$, $\alpha_{i_1}\rangle$ to the responder $U_{j}$.\\
\noindent \textbf{(b)}~After receiving the message $\langle id_{i}$,
$x_{i}$, $w_{i_1}$, $\alpha_{i_1}\rangle$ from $U_{i}$, the
responder $U_{j}$ draws two samples $r_{j}$,
$f_{j}$$\xleftarrow[]{R}\chi_{\beta}$ and computes the followings:
\begin{itemize}
\item[]$x_{j} = \textbf{\emph{a}}\cdot r_{j} + 2\cdot\textbf{\emph{f}}_{j}$
\item[]$t_{j_1} = s_{j}\cdot p_{i}$
\item[]$w_{j_1} = \mbox{\textbf{Cha}}(t_{j_1})$
\item[]$\sigma_{j_1} = \mbox{\textbf{Mod}}_{2}(t_{j_1}, w_{j_1})$
\item[]$\alpha_{j_1} = H_{1}(id_{j}, x_{j}, \sigma_{j_1})$
\end{itemize}
\noindent $U_{j}$ sends the message $\langle id_{j}$, $x_{j}$,
$w_{j_1}$, $\alpha_{j_1}\rangle$ to $U_{i}$.\\
\noindent\textbf{(c)}~After receiving the message $\langle id_{j}$,
$x_{j}$, $w_{j_1}$, $\alpha_{j_1}\rangle$ from $U_{j}$, $U_{i}$
further computes the followings:
\begin{itemize}
\item[]$\sigma_{j_1}^{\prime} = \mbox{\textbf{Mod}}_{2}(t_{i_1}, w_{j_1})$
\item[]$\alpha_{j_1}^{\prime} = H_{1}(id_{j}, x_{j}, \sigma_{j_1}^{\prime})$
\end{itemize}
\noindent If $\alpha_{j_1}^{\prime} = \alpha_{j_1}$, $U_{i}$
confirms that the message $\langle id_{j}$, $x_{j}$, $w_{j_1}$,
$\alpha_{j_1}\rangle$ is correctly received from $U_{j}$. User
$U_{i}$ further computes the followings:
\begin{itemize}
\item[]$t_{i_2} = r_{i}\cdot x_{j}$
\item[]$w_{i_2} = \mbox{\textbf{Cha}}(t_{i_2})$
\item[]$\sigma_{i_2} = \mbox{\textbf{Mod}}_{2}(t_{i_2}, w_{i_2})$
\item[]$\alpha_{i_2} = H_{2}(id_{i}, x_{i}, x_{j}, \sigma_{i_2})$
\end{itemize}
\noindent $U_{i}$ sends the message $\langle id_{i}$, $w_{i_2}$,
$\alpha_{i_2}\rangle$ to $U_{j}$.\\
\noindent\textbf{(d)}~After receiving the message $\langle id_{i}$,
$w_{i_2}$, $\alpha_{i_2}\rangle$ from $U_{i}$, $U_{j}$ computes
\begin{itemize}
\item[]$\sigma_{i_1}^{\prime} = \mbox{\textbf{Mod}}_{2}(t_{j_1}$, $w_{i_1})$
\item[]$\alpha_{i_1}^{\prime} = H_{2}(id_{i}, x_{i}, \sigma_{i_1}^{\prime})$
\end{itemize}
\noindent If $\alpha_{i_1}^{\prime} = \alpha_{i_1}$, $U_{j}$
confirms that the message $\langle id_{i}$, $x_{i}$, $w_{i_1}$,
$\alpha_{i_1}\rangle$ is correctly received from $U_{i}$. $U_{j}$
further computes the followings:
\begin{itemize}
\item[]$t_{j_2} = r_{j}\cdot x_{i}$
\item[]$w_{j_2} = \mbox{\textbf{Cha}}(t_{j_2})$
\item[]$\sigma_{j_2} = \mbox{\textbf{Mod}}_{2}(t_{j_2}, w_{j_2})$
\item[]$\alpha_{j_2} = H_{2}(id_{j}, x_{j}, x_{i}, \sigma_{j_2})$
\end{itemize}
\noindent $U_{j}$ sends the message $\langle id_{j}$, $w_{j_2}$,
$\alpha_{j_2}\rangle$ to $U_{i}$. \\
\noindent After receiving the message $\langle id_{j}$, $w_{j_2}$,
$\alpha_{j_2}\rangle$ from $U_{j}$, $U_{i}$ computes
\begin{itemize}
\item[]$\sigma_{j_2}^{\prime} = \mbox{\textbf{Mod}}_{2}(t_{i_2}, w_{j_2})$
\item[]$\alpha_{j_2}^{\prime} = H_{2}(id_{j}, x_{j}, x_{i}, \sigma_{j_2}^{\prime})$
\end{itemize}
\noindent If $\alpha_{j_2}^{\prime} = \alpha_{j_2}$, then $U_{i}$
computes the session identifier as $sid = (id_{i}$, $id_{j}$,
$x_{i}$, $x_{j}$, $w_{i_1}$, $w_{j_1}$, $\alpha_{i_1}$,
$\alpha_{j_1})$ and the session key as $SK = H_{3}(sid$,
$\sigma_{i_1}$, $\sigma_{i_2}$, $\sigma_{j_1}^{\prime}$,
$\sigma_{j_2}^{\prime})$.\\
\noindent After receiving the message $\langle id_{i}$, $w_{i_2}$,
$\alpha_{i_2}\rangle$ from $U_{i}$, $U_{j}$ computes
\begin{itemize}
\item[] $\sigma_{i_2}^{\prime} = \mbox{\textbf{Mod}}_{2}(t_{j_2}, w_{i_2})$
\item[] $\alpha_{i_2}^{\prime} = H_{2}(id_{i}, x_{i}, x_{j}, \sigma_{i_2}^{\prime})$
\end{itemize}
\noindent If $\alpha_{i_2}^{\prime} = \alpha_{i_2}$, then $U_{j}$
computes the session identifier as $sid = (id_{i}$, $id_{j}$,
$x_{i}$, $x_{j}$, $w_{i_1}$, $w_{j_1}$, $\alpha_{i_1}$,
$\alpha_{j_1})$ and the session key as $SK = H_{3}(sid$,
$\sigma_{i_1}^{\prime}$, $\sigma_{i_2}^{\prime}$, $\sigma_{j_1}$,
$\sigma_{j_2})$.

The key agreement phase of our LB-2PAKA protocol is illustrated in
the Figure~\ref{F1}.
%%%%%%%%%%%%%%%%%%%%%%%%%%%%%%%%%%%%%%%%%%%%%%%%%%%%%%%%%%%%%%%%%%%%%%%%%%%%%%%%%%%%%%%%%%%%%%%%%%%%%%%%%%%%%%%%%%%%%%%%%%%%%%%%%%%%%%%%%%%%%%%%%%%%%%%%%%%%%%%%%
\begin{figure*}[t]
\centering \scalebox{0.85} {
\begin{tabular}{|l l l|}
\hline
~~~~~\textbf{User~$U_{i}$ (Initiator)}                                                                                                          &~~Open network         &~~~~~~~~~~\textbf{User~$U_{j}$ (Responder)}  \\
\hline
Sample $r_{i}$, $f_{i}$ $\xleftarrow[]{R}\chi_{\beta}$                                                                          &         & Sample $r_{j}$, $f_{j}$ $\xleftarrow[]{R}\chi_{\beta}$   \\

Compute $x_{i} = \textbf{\emph{a}}\cdot r_{i} + 2\cdot\textbf{\emph{f}}_{i}$, $t_{i_1} = s_{i}\cdot p_{j}$                      &         & Compute $x_{j} = \textbf{\emph{a}}\cdot r_{j} + 2\cdot\textbf{\emph{f}}_{j}$, $t_{j_1} = s_{j}\cdot p_{i}$ \\
Compute $w_{i_1} = \mbox{\textbf{Cha}}(t_{i_1})$, $\sigma_{i_1} = \mbox{\textbf{Mod}}_{2}(t_{i_1}, w_{i_1})$                    &         & Compute $w_{j_1} = \mbox{\textbf{Cha}}(t_{j_1})$, $\sigma_{j_1} = \mbox{\textbf{Mod}}_{2}(t_{j_1}, w_{j_1})$   \\
Compute $\alpha_{i_1} = H_{1}(id_{i}, x_{i}, \sigma_{i_1})$                                                                     &         & Compute $\alpha_{j_1} = H_{1}(id_{j}, x_{j}, \sigma_{j_1})$ \\
                         & $\underrightarrow{~~\langle id_{i}, x_{i}, w_{i_1}, \alpha_{i_1}\rangle~~}$ &   \\
                         & $\underleftarrow{~~\langle id_{j}, x_{j}, w_{j_1}, \alpha_{j_1}\rangle~~}$ &   \\
Compute $\sigma_{j_1}^{\prime} = \mbox{\textbf{Mod}}_{2}(t_{i_1}, w_{j_1})$, $\alpha_{j_1}^{\prime} = H_{1}(id_{j}, x_{j}, \sigma_{j_1}^{\prime})$  &         & Compute $\sigma_{i_1}^{\prime} = \mbox{\textbf{Mod}}_{2}(t_{j_1}, w_{i_1})$, $\alpha_{i_1}^{\prime} = H_{1}(id_{i}, x_{i}, \sigma_{i_1}^{\prime})$ \\
If $\alpha_{j_1}^{\prime}\neq \alpha_{j_1}$, abort                                                                               &         & If $\alpha_{i_1}^{\prime}\neq \alpha_{i_1}$, abort  \\
Else, compute $t_{i_2} = r_{i}\cdot x_{j}$, $w_{i_2} = \mbox{\textbf{Cha}}(t_{i_2})$                                             &         & Else, compute $t_{j_2} = r_{j}\cdot x_{i}$, $w_{j_2} = \mbox{\textbf{Cha}}(t_{j_2})$ \\
Compute $\sigma_{i_2} = \mbox{\textbf{Mod}}_{2}(t_{i_2}, w_{i_2})$, $\alpha_{i_2} = H_{2}(id_{i}, x_{i}, x_{j}, \sigma_{i_2})$   &         & Compute $\sigma_{j_2} = \mbox{\textbf{Mod}}_{2}(t_{j_2}, w_{j_2})$, $\alpha_{j_2} = H_{2}(id_{j}, x_{j}, x_{i}, \sigma_{j_2})$\\
                         & $\underrightarrow{~~\langle id_{i}, w_{i_2}, \alpha_{i_2}\rangle~~}$ &   \\
                         & $\underleftarrow{~~\langle id_{j}, w_{j_2}, \alpha_{j_2}\rangle~~}$ &   \\
Compute $\sigma_{j_2}^{\prime} = \mbox{\textbf{Mod}}_{2}(t_{i_2}, w_{j_2})$, $\alpha_{j_2}^{\prime} = H_{2}(id_{j}, x_{j}, x_{i}, \sigma_{j_2}^{\prime})$  &         & Compute $\sigma_{i_2}^{\prime} = \mbox{\textbf{Mod}}_{2}(t_{j_2}, w_{i_2})$, $\alpha_{i_2}^{\prime} = H_{2}(id_{i}, x_{i}, x_{j}, \sigma_{i_2}^{\prime})$ \\
If $\alpha_{j_2}^{\prime}\neq \alpha_{j_2}$, abort                                                                      &         & If $\alpha_{i_2}^{\prime}\neq \alpha_{i_2}$, abort\\
Else, compute $sid = (id_{i}$, $id_{j}$, $x_{i}$, $x_{j}$, $w_{i_1}$, $w_{j_1}$, $\alpha_{i_1}$, $\alpha_{j_1})$        &         & Else, compute $sid = (id_{i}$, $id_{j}$, $x_{i}$, $x_{j}$, $w_{i_1}$, $w_{j_1}$, $\alpha_{i_1}$, $\alpha_{j_1})$\\
Compute session key $SK = H_{3}(sid$, $\sigma_{i_1}$, $\sigma_{i_2}$, $\sigma_{j_1}^{\prime}$, $\sigma_{j_2}^{\prime})$ &         & Compute session key $SK = H_{3}(sid$, $\sigma_{i_1}^{\prime}$, $\sigma_{i_2}^{\prime}$, $\sigma_{j_1}$, $\sigma_{j_2})$\\
\hline
\end{tabular}}
\vspace{0.3cm} \caption{Authenticated key agreement phase}
\label{F1}
\end{figure*}
%%%%%%%%%%%%%%%%%%%%%%%%%%%%%%%%%%%%%%%%%%%%%%%%%%%%%%%%%%%%%%%%%%%%%%%%%%%%%%%%%%%%%%%%%%%%%%%%%%%%%%%%%%%%%%%%%%%%%%%%%%%%%%%%%%%%%%%%%%%%%%%%%%%%%%%%%%%%%%%%%


\section{Security Analysis}
\label{L5}
\subsection{Correctness}
\begin{theorem}
\label{Th1} \normalfont Let $q>16\beta^{2}n^{\frac{3}{2}}$ be an odd
prime number. Let $U_{i}$ and $U_{j}$ follow the proposed LB-2PAKA
protocol, then $U_{i}$ and $U_{j}$ will successfully end with a
common session key $SK = H_{3}(sid$, $\sigma_{i_1}$, $\sigma_{i_2}$,
$\sigma_{j_1}$, $\sigma_{j_2})$.
\end{theorem}

\begin{proof}
\normalfont In the authenticated key agreement phase of our LB-2PAKA
protocol, user $U_{i}$ justifies the legitimacy of $U_{j}$ by
verifying whether $\alpha_{j_1}^{\prime} = \alpha_{j_1}$ and
$\alpha_{j_2}^{\prime} = \alpha_{j_2}$. Since:
\begin{eqnarray}
\label{Gx1} t_{i_1} = s_{i}\cdot p_{j} = a\cdot s_{i}\cdot s_{j} +
2\cdot s_{i}\cdot e_{j}
\end{eqnarray}
\begin{eqnarray}
\label{Gx2} t_{j_1} = s_{j}\cdot p_{i} = a\cdot s_{i}\cdot s_{j} +
2\cdot s_{j}\cdot e_{i}
\end{eqnarray}
Subtracting the Eq.~(\ref{Gx2}) from the Eq.~(\ref{Gx1}), we get:
\begin{eqnarray}
\label{Gx3} t_{i_1} - t_{j_1} = 2\cdot s_{i}\cdot e_{j} - 2\cdot s_{j}\cdot e_{i}
\end{eqnarray}

Based on the Lemma~\ref{LM1} and Lemma~\ref{LM2}, we have:
\begin{eqnarray}
\label{Gx4} ||s_{i}\cdot e_{j} - s_{j}\cdot e_{i}||_{2} \nonumber
&\leq& ||s_{i}\cdot e_{j}||_{2} + ||s_{j}\cdot e_{i}||_{2} \nonumber \\
&<& \sqrt{n}\cdot||s_{i}||_{2}\cdot||e_{j}||_{2} + \sqrt{n}\cdot||s_{j}||_{2}\cdot||e_{i}||_{2} \nonumber \\
&<& \sqrt{n}\cdot\beta\cdot\sqrt{n}\cdot\beta\cdot\sqrt{n} + \sqrt{n}\cdot\beta\cdot\sqrt{n}\cdot\beta\cdot\sqrt{n} \nonumber \\
&=& 2\cdot\beta^{2}\cdot n^{\frac{3}{2}} \nonumber \\
&<& \frac{q}{8}~~[\mbox{Since}~q>16\beta^{2}n^{\frac{3}{2}}]
\end{eqnarray}

Hence, from the Lemma~\ref{LM3} and Eq.~(\ref{Gx4}), we get:
\begin{eqnarray}
\label{Gx5}
\sigma_{j_1}^{\prime}\nonumber
&=& \mbox{\textbf{Mod}}_{2}(t_{i_1}, w_{j_1}) \nonumber \\
&=& \mbox{\textbf{Mod}}_{2}(t_{i_1}, \mbox{\textbf{Cha}}(t_{j_1})) \nonumber \\
&=& \mbox{\textbf{Mod}}_{2}(t_{j_1}, \mbox{\textbf{Cha}}(t_{j_1})) \nonumber \\
&=& \sigma_{j_1}
\end{eqnarray}
Thus, according to the Eq.~(\ref{Gx5}), we also get:
\begin{eqnarray}
\label{Gx6}
\alpha_{j_1}^{\prime} \nonumber
&=& H_{1}(id_{j}, x_{j}, \sigma_{j_1}^{\prime}) \nonumber \\
&=& H_{1}(id_{j}, x_{j}, \sigma_{j_1}) \nonumber \\
&=& \alpha_{j_1}
\end{eqnarray}
Again, we have:
\begin{eqnarray}
\label{Gx7} t_{i_2} = x_{i}\cdot x_{j} = a\cdot r_{i}\cdot r_{j} +
2\cdot r_{i}\cdot f_{j}
\end{eqnarray}
\begin{eqnarray}
\label{Gx8} t_{j_2} = x_{j}\cdot x_{i} = a\cdot r_{i}\cdot r_{j} +
2\cdot x_{j}\cdot f_{i}
\end{eqnarray}
Subtracting the Eq.~(\ref{Gx8}) from the Eq.~(\ref{Gx7}), we get:
\begin{eqnarray}
\label{Gx9} t_{i_2} - t_{j_2} = 2\cdot r_{i}\cdot f_{j} - 2\cdot
r_{j}\cdot f_{i}
\end{eqnarray}

From the Lemma~\ref{LM1} and Lemma~\ref{LM2}, we get:
\begin{eqnarray}
\label{Gx9} ||r_{i}\cdot f_{j} - r_{j}\cdot f_{i}||_{2} <
\frac{q}{8}
\end{eqnarray}

Hence, from the Lemma~\ref{LM3} and Eq.~(\ref{Gx9}), we have:
\begin{eqnarray}
\label{Gx10}
\sigma_{j_2}^{\prime}\nonumber
&=& \mbox{\textbf{Mod}}_{2}(t_{i_2}, w_{j_2}) \nonumber \\
&=& \mbox{\textbf{Mod}}_{2}(t_{i_2}, \mbox{\textbf{Cha}}(t_{j_2})) \nonumber \\
&=& \mbox{\textbf{Mod}}_{2}(t_{j_2}, \mbox{\textbf{Cha}}(t_{j_2})) \nonumber \\
&=& \sigma_{j_2}
\end{eqnarray}
Thus, according to the Eq.~(\ref{Gx10}), we also get:
\begin{eqnarray}
\label{Gx11}
\alpha_{j_2}^{\prime} \nonumber
&=& H_{2}(id_{j}, x_{j}, x_{i}, \sigma_{j_2}^{\prime})\nonumber \\
&=& H_{2}(id_{j}, x_{j}, x_{i}, \sigma_{j_2}) \nonumber \\
&=& \alpha_{j_2}
\end{eqnarray}

According to the Eq.~(\ref{Gx5}), Eq.~(\ref{Gx11}) and $sid =
(id_{i}$, $id_{j}$, $x_{i}$, $x_{j}$, $w_{i_1}$, $w_{j_1}$,
$\alpha_{i_1}$, $\alpha_{j_1})$, user $U_{i}$ computes the session
key as
\begin{eqnarray}
\label{Gx12} SK = H_{3}(sid, \sigma_{i_1}, \sigma_{i_2},
\sigma_{j_1}^{\prime}, \sigma_{j_2}^{\prime}) = H_{3}(sid,
\sigma_{i_1}, \sigma_{i_2}, \sigma_{j_1}, \sigma_{j_2})
\end{eqnarray}
User $U_{j}$ justifies the legitimacy of $U_{i}$ by verifying
whether $\alpha_{i_1}^{\prime} = \alpha_{i_1}$ and
$\alpha_{i_2}^{\prime} = \alpha_{i_2}$. Based on the Lemma~\ref{LM3}
and Eq.~(\ref{Gx6}), we get:
\begin{eqnarray}
\label{Gx13}
\sigma_{i_1}^{\prime}\nonumber
&=& \mbox{\textbf{Mod}}_{2}(t_{j_1}, w_{i_1}) \nonumber \\
&=& \mbox{\textbf{Mod}}_{2}(t_{j_1}, \mbox{\textbf{Cha}}(t_{i_1})) \nonumber \\
&=& \mbox{\textbf{Mod}}_{2}(t_{i_1}, \mbox{\textbf{Cha}}(t_{i_1})) \nonumber \\
&=& \sigma_{i_1}
\end{eqnarray}
Thus, according to the Eq.~(\ref{Gx13}), we also get:
\begin{eqnarray}
\label{Gx14} \alpha_{i_1}^{\prime} \nonumber
&=& H_{1}(id_{i}, x_{i}, \sigma_{i_1}^{\prime}) \nonumber \\
&=& H_{1}(id_{i}, x_{i}, \sigma_{i_1}) \nonumber \\
&=& \alpha_{i_1}
\end{eqnarray}
Hence, from the Lemma~\ref{LM3} and Eq.~(\ref{Gx9}), we get:
\begin{eqnarray}
\label{Gx15} \sigma_{i_2}^{\prime}\nonumber
&=& \mbox{\textbf{Mod}}_{2}(t_{j_2}, w_{i_2}) \nonumber \\
&=& \mbox{\textbf{Mod}}_{2}(t_{j_2}, \mbox{\textbf{Cha}}(t_{i_2})) \nonumber \\
&=& \mbox{\textbf{Mod}}_{2}(t_{i_2}, \mbox{\textbf{Cha}}(t_{i_2})) \nonumber \\
&=& \sigma_{i_2}
\end{eqnarray}
Thus, according to the Eq.~(\ref{Gx15}), we also get:
\begin{eqnarray}
\label{Gx16} \alpha_{i_2}^{\prime} \nonumber
&=& H_{2}(id_{i}, x_{i}, x_{j}, \sigma_{i_2}^{\prime})\nonumber \\
&=& H_{2}(id_{i}, x_{i}, x_{j}, \sigma_{i_2})\nonumber \\
&=& \alpha_{i_2}
\end{eqnarray}

According to the Eq.~(\ref{Gx13}), Eq.~(\ref{Gx15}) and $sid =
(id_{i}$, $id_{j}$, $x_{i}$, $x_{j}$, $w_{i_1}$, $w_{j_1}$,
$\alpha_{i_1}$, $\alpha_{j_1})$, $U_{j}$ calculates the session key
as
\begin{eqnarray}
\label{Gx17} SK = H_{3}(sid, \sigma_{i_1}^{\prime},
\sigma_{i_2}^{\prime}, \sigma_{j_1}, \sigma_{j_2}) H_{3}(sid,
\sigma_{i_1}, \sigma_{i_2}, \sigma_{j_1}, \sigma_{j_2})
\end{eqnarray}
According to the Eq.~(\ref{Gx12}) and Eq.~(\ref{Gx17}), $U_{i}$ and
$U_{j}$ successfully established a common secret session key $SK =
H_{3}(sid, \sigma_{i_1}, \sigma_{i_2}, \sigma_{j_1}, \sigma_{j_2})$
between them.
\end{proof}

\subsection{Provable Security}
\begin{theorem}
\label{Th2} \normalfont Let $Adv^{LB-2PAKA}_{\mathcal{A}}(t)$ be the
probability of success of $\mathcal{A}$ to infringement the semantic
security of the session key and mutual authentication between
$U_{i}$ and $U_{j}$ in our LB-2PAKA protocol with in a polynomial
time-bound $t$. Assume that $Adv^{RLWE}_{\mathcal{A}}(t^{\prime})$
is the advantage of $\mathcal{A}$ to solve an instance of the RLWE
problem within a polynomial time-bound $t^{\prime}$. $\mathcal{A}$
is allowed to ask maximum $q_{H_{i}}$ ($i = 1, 2, 3$), $q_{se}$, and
$q_{ex}$ number of queries to $H_{i}$ oracle, \emph{Send} oracle,
and \emph{Execute} oracle, respectively and $T_{smul}$ denotes the
running time required to perform one componentwise multiplication
with scalar operation in $\mathbb{R}_{q}$. Therefore we have:
\begin{eqnarray}
Adv^{LB-2PAKA}_{\mathcal{A}}(t) & \leq& \frac{q_{H_1}^{2}}{2l} +
\frac{q_{H_2}^{2}}{2l} + \frac{(q_{ex}+q_{se})^{2}}{2^{l}} +
\frac{q_{H_3}^{2}}{2^{l}}+(q_{ex}+q_{se})Adv^{RLWE}_{\mathcal{A}}(t^{\prime})
\nonumber
\end{eqnarray}
\begin{eqnarray}
t^{\prime}&\leq& t + (2q_{ex} + 4q_{sc} + 2q_{H_1} + 2q_{H_2} +
1)\cdot T_{smul} \nonumber
\end{eqnarray}
\end{theorem}

\begin{proof}
\normalfont Assume that the semantic security of the session key of
our LB-2PAKA protocol can be violated by $\mathcal{A}$, then a
challenger $\mathcal{C}$, which runs a PPT algorithm, can solve an
instance of the RLWE problem in
$\mathbb{R}_{q}\times\mathbb{R}_{q}$. We consider the games
\textbf{$Game_i$} $(0\le i\le 4)$, where \textbf{$Game_0$} defines
the real attack on our LB-2PAKA protocol \cite{islam2016,VS2006}. In
each game \textbf{$Game_i$} is interactively played between
$\mathcal{A}$ and $\mathcal{C}$. In each game, to get a response
with the correct output for an oracle query, $\mathcal{C}$ runs a
PPT algorithm. It can be noted that, for every \textbf{$Game_i$}, an
event $X_{i}$ $(0\le i\le 4)$ is defined based on which
$\mathcal{A}$ wins in breaching the semantic security a session key
of our LB-2PAKA protocol in $Game_i$. Suppose that the event $E$
that is independent of $X_i$, may occur during the simulation of the
LB-2PAKA protocol by $\mathcal{A}$ such that $E$ is detectable by
$\mathcal{C}$. Noted that $Game_i$ and $Game_{i+1}$ are
indistinguishable unless $E$ occurs. Therefore, we have:
\begin{eqnarray}
|Pr[X_{i+1}]-Pr[X_{i}]| &\leq & Pr[E]
\end{eqnarray}

\noindent $Game_0$:~The simulation of this game is indistinguishable
to the real attack on our LB-2PAKA protocol in ROM. In this game,
$\mathcal{A}$ does not alter the simulation strategies of the
different oracles, i.e., all the instances of the users
$P\in\mathbb{U}$ is modeled as the real execution of the LB-2PAKA
protocol in ROM. Therefore, we have:
\begin{eqnarray}
\label{G0} Adv^{LB-2PAKA}_{\mathcal{A}}(t)=|Pr[X_0] - \frac{1}{2}|
\end{eqnarray}

\noindent $Game_1$:~The simulation of this game is indistinguishable
with the game $Game_0$ other than $\mathcal{A}$ executes different
\textbf{Hash queries} other than the ways executed in the game
$Game_0$. In this game, $\mathcal{C}$ maintains three
initially-empty lists $H^{list}_{j}$, $j=1,2,3$ with the tuples of
the form $(x$, $y)$, where $x$ is the input to a \textbf{Hash}
oracle and $y$ is the corresponding output. If $\mathcal{A}$ asks a
\textbf{Hash} query for the input $x$, $\mathcal{C}$ explores
$H^{list}_{j}$, $j=1,2,3$ and returns $y$ as answer to $\mathcal{A}$
if a record $(x, y)$ is found in $H^{list}_{j}$, $j=1,2,3$.
Otherwise, $\mathcal{C}$ randomly chooses a value
$y\in\mathbb{Z}_{q}$, returns it to $\mathcal{A}$ and incorporates
the new tuple $(x$, $y)$ to $H^{list}_{j}$, $j=1,2,3$. The real
attack in ROM and the simulation of this game are absolutely
indistinguishable since the other oracles \textbf{Execute},
\textbf{Send}, \textbf{Reveal}, \textbf{RevealEph},
\textbf{Corrupt}, and \textbf{Test} are also simulated similar way
as executed in the real attack. Thus, we have:
\begin{eqnarray}
\label{G1} Pr[X_1]=Pr[X_0]
\end{eqnarray}

\noindent $Game_2$:~The simulation of this game is indistinguishable
with the game $Game_1$ other than it will be aborted if a
\emph{collision} is found in the simulation of $\langle id_{i}$,
$x_{i}$, $w_{i_1}$, $\alpha_{i_1}\rangle$, $\langle id_{j}$,
$x_{j}$, $w_{j_1}$, $\alpha_{j_1}\rangle$, $\langle id_{i}$,
$w_{i_2}$, $\alpha_{i_2}\rangle$, and $\langle id_{j}$, $w_{j_2}$,
$\alpha_{j_2}\rangle$. Based on the birthday paradox, the
probability of collisions in output of the oracles $H_{1}$ and
$H_{2}$ are at most $\frac{q_{H_1}^{2}}{2l}$ and
$\frac{q_{H_2}^{2}}{2l}$, respectively. Likewise, the probability of
collision in output of the oracle $H_{3}$ is at most $\frac{(q_{sc}
+ q_{ex})^{2}}{2^{l}}$. Thus, we have:
\begin{eqnarray}
\label{G2} |Pr[X_2]-Pr[X_1]| &\leq&
\frac{q_{H_1}^{2}}{2l}+\frac{q_{H_2}^{2}}{2l}+\frac{(q_{ex}+q_{se})^{2}}{2^{l}}
\end{eqnarray}

\noindent $Game_3$:~The simulation of this game is indistinguishable
with the game $Game_2$ other than $\mathcal{A}$ queries oracle
$H_{3}$ with the input of $(sid$, $\sigma_{i_1}$, $\sigma_{i_2}$,
$\sigma_{j_1}^{\prime}$, $\sigma_{j_2}^{\prime})$. Therefore, the
probability of guessing the bit $b$, which was selected in the
\textbf{\emph{Test}} query with the oracle $H_{3}$ is at most
$\frac{q_{H_3}^{2}}{2^{l}}$. Thus, we have:

\begin{eqnarray}
\label{G3} |Pr[X_3]-Pr[X_2]| &\leq& \frac{q_{H_3}^{2}}{2^{l}}
\end{eqnarray}
\noindent $Game_4$: In this game, different types of guessing
attacks through offline modes are considered, i.e., he session key
$SK$ may be computed without simulating $H_{3}$ oracle.
\begin{itemize}
\item Here, $SK$ is predicted without simulating the oracle $H_{3}$,
i.e., $SK$ is absolutely independent from $H_{3}$ and $(sid$,
$\sigma_{i_1}$, $\sigma_{i_2}$, $\sigma_{j_1}^{\prime}$,
$\sigma_{j_2}^{\prime})$. According to our LB-2PAKA protocol, both
the usres$U_{i}$ and $U_{j}$ compute the session key as $SK =
H_{3}(sid$, $\sigma_{i_1}$, $\sigma_{i_2}$, $\sigma_{j_1}^{\prime}$,
$\sigma_{j_2}^{\prime})$, where $sid = (id_{i}$, $id_{j}$, $x_{i}$,
$x_{j}$, $w_{i_1}$, $w_{j_1}$, $\alpha_{i_1}$, $\alpha_{j_1})$,
$\sigma_{i_1} = \mbox{\textbf{Mod}}_{2}(t_{i_1}, w_{i_1})$,
$\sigma_{j_1} = \mbox{\textbf{Mod}}_{2}(t_{j_1}, w_{j_1})$,
$\sigma_{i_2} = \mbox{\textbf{Mod}}_{2}(t_{i_2}, w_{i_2})$ and
$\sigma_{j_2} = \mbox{\textbf{Mod}}_{2}(t_{j_2}, w_{j_2})$.
Accordingly, we say that if $\mathcal{A}$ can successfully guess
$SK$, then the adversary $\mathcal{A}$ can solve an instance of RLWE
problem with in polynomial time bound $t^{\prime}$ by interacting
with the challenger $\mathcal{C}$. Therefore, we have:
\begin{eqnarray}
\label{G4} |Pr[X_4]-Pr[X_3]| &\leq &
(q_{ex}+q_{se})Adv^{RLWE}_{\mathcal{A}}(t^{\prime})
\end{eqnarray}

\item If $\mathcal{A}$ doesn't simulate $H_{3}$ with the correct
input $(sid$,$\sigma_{A}$, $\sigma_{B}$), then $\mathcal{A}$ has no
advantage in distinguishing the real session key and a random
session key
\end{itemize}
Therefore, we have:
\begin{eqnarray}
\label{G5} Pr[X_4] = \frac{1}{2}
\end{eqnarray}
From the Eq. (\ref{G0}), we have:
\begin{eqnarray}
\label{G6} Adv^{LB-2PAKA}_{\mathcal{A}}(t) \nonumber
&=& |Pr[X_0] - \frac{1}{2}| \nonumber \\
&=& |Pr[X_0] - Pr[X_4] + Pr[X_4] - \frac{1}{2}| \nonumber \\
&=& |Pr[X_0] - Pr[X_4]|
\end{eqnarray}

From the Eq. (\ref{G6}) and using the Eqs. (\ref{G0}) - (\ref{G5}),
we have:
\begin{eqnarray}
Adv^{LB-2PAKA}_{\mathcal{A}}(t)\nonumber
&=& |Pr[X_0] - Pr[X_1] + Pr[X_1] - Pr[X_2] + Pr[X_2] - Pr[X_3] + Pr[X_3] - Pr[X_4]| \nonumber \\
& \leq & \frac{q_{H_1}^{2}}{2l} + \frac{q_{H_2}^{2}}{2l} +
\frac{(q_{ex}+q_{se})^{2}}{2^{l}} +
\frac{q_{H_3}^{2}}{2^{l}}+(q_{ex}+q_{se})Adv^{RLWE}_{\mathcal{A}}(t^{\prime})
\nonumber
\end{eqnarray}

\begin{eqnarray}
t^{\prime} &\leq& t + (2q_{ex} + 4q_{sc} + 2q_{H_1} + 2q_{H_2} +
1)\cdot T_{smul} \nonumber
\end{eqnarray}
\end{proof}

\subsection{Further Security Discussion}
Ths subsection is proposed to justify the security attributes
\cite{Islamgpwpc2015} provided by our LB-2PAKA protocol.

\subsubsection{\textbf{Man-in-the-Middle (MITM) Attack}}~In our LB-2PAKA
protocol, user $U_{i}$ sends the messages $\langle id_{i}$, $x_{i}$,
$w_{i_1}$, $\alpha_{i_1}\rangle$ and $\langle id_{i}$, $w_{i_2}$,
$\alpha_{i_2}\rangle$ to the user $U_{j}$, where $\alpha_{i_1} =
H_{1}(id_{i}, x_{i}, \sigma_{i_1})$ and $\alpha_{i_2} =
H_{2}(id_{i}, x_{i}, x_{j}, \sigma_{i_2})$. Assume that an adversary
$\mathcal{A}$ captures these two messages and then tried to
fabricated them. However, the fabrication of the messages $\langle
id_{i}$, $x_{i}$, $w_{i_1}$, $\alpha_{i_1}\rangle$ and $\langle
id_{i}$, $w_{i_2}$, $\alpha_{i_2}\rangle$ is not possible. Because
$\mathcal{A}$ has to compute two forged authenticators in place of
$\alpha_{i_1}$ and $\alpha_{i_2}$. For this purpose, $\mathcal{A}$
has to compute $w_{i_1} = \mbox{\textbf{Cha}}(t_{i_1})$ and
$\sigma_{i_1} = \mbox{\textbf{Mod}}_{2}(t_{i_1}, w_{i_1})$, which is
equivalent solve the RLWR problem for the instance
$(\textbf{\emph{a}}$, $p_{i}) = (\textbf{\emph{a}}$,
$\textbf{\emph{a}}\cdot s_{i} + 2\cdot\textbf{\emph{e}}_{i})$ or
$(\textbf{\emph{a}}$, $p_{j}) = (\textbf{\emph{a}}$,
$\textbf{\emph{a}}\cdot s_{j} + 2\cdot\textbf{\emph{e}}_{j})$.
Similarly, $\mathcal{A}$ has to solve the RLWE problem for the
instance $(\textbf{\emph{a}}$, $x_{i}) = (\textbf{\emph{a}}$,
$\textbf{\emph{a}}\cdot r_{i} + 2\cdot\textbf{\emph{f}}_{i})$ or
$(\textbf{\emph{a}}$, $x_{j}) = (\textbf{\emph{a}}$,
$\textbf{\emph{a}}\cdot r_{j} + 2\cdot\textbf{\emph{f}}_{j})$ to
compute the forged authenticator in place of $\alpha_{i_2}$.
Therefore, $\mathcal{A}$ cannot impersonate the initiator $U_{i}$.
In the same way, user $U_{j}$ sends the messages $\langle id_{j}$,
$x_{j}$, $w_{j_1}$, $\alpha_{j_1}\rangle$ and $\langle id_{j}$,
$w_{j_2}$, $\alpha_{j_2}\rangle$ to the user $U_{i}$, where
$\alpha_{j_1} = H_{1}(id_{j}, x_{j}, \sigma_{j_1})$ and
$\alpha_{j_2} = H_{2}(id_{j}, x_{j}, x_{i}, \sigma_{j_2})$. By the
similar reason, $\mathcal{A}$ cannot impersonate the responder
$U_{j}$. Accordingly, our LB-2PAKA protocol withstands the
Man-in-the-Middle (MITM) attack.

\subsubsection{\textbf{Known-Key (KK) Attack}}~The known-key attack
says that an adversary $\mathcal{A}$ is unable to generate the
current session key even if the exchanged messages and the session
keys of some earlier sessions are known to $\mathcal{A}$. In our
LB-2PAKA protocol, $U_{i}$ and $U_{j}$ calculates the session $SK =
H_{3}(sid$, $\sigma_{i_1}$, $\sigma_{i_2}$, $\sigma_{j_1}$,
$\sigma_{j_2})$ in a session. Assume that $\mathcal{A}$ captures all
the messages $\langle id_{i}$, $x_{i}$, $w_{i_1}$,
$\alpha_{i_1}\rangle$, $\langle id_{i}$, $w_{i_2}$,
$\alpha_{i_2}\rangle$, $\langle id_{j}$, $x_{j}$, $w_{j_1}$,
$\alpha_{j_1}\rangle$ and $\langle id_{j}$, $w_{j_2}$,
$\alpha_{j_2}\rangle$ for a session and the session key $SK$ of that
session is disclosed. Despite the disclosure of $SK$, $\mathcal{A}$
is unable compute the session keys of other sessions due to the
onewayness of the hash function $H_{3}(\cdot)$ and the freshness of
the samples $r_{i}$, $r_{j}$, $f_{i}$ and $f_{j}$ drawn at random
from $\chi_{\beta}$.

\subsubsection{\textbf{Key-Compromise Impersonation (KCI) Attack}}
Assume that an adversary $\mathcal{A}$ knows the private key
$(s_{i}$, $e_{i})$ of the initiator $U_{i}$. However, $\mathcal{A}$
may not be able to impersonate the responder $U_{j}$ to $U_{i}$. In
a session, $U_{i}$ draws two random samples $r_{i}$,
$f_{i}\xleftarrow[]{R}\chi_{\beta}$, then computes the message
$\langle id_{i}$, $x_{i}$, $w_{i_1}$, $\alpha_{i_1}\rangle$ and
sends it to $U_{j}$ ($\mathcal{A}$). Assume that $\mathcal{A}$
captures the message $\langle id_{i}$, $x_{i}$, $w_{i_1}$,
$\alpha_{i_1}\rangle$ and tried to impersonate $U_{j}$ to $U_{i}$ as
follows. To impersonate $U_{j}$ and to know the session key $SK$ of
that session, $\mathcal{A}$ draws two random samples $r_{j}$,
$f_{j}\xleftarrow[]{R}\chi_{\beta}$, computes the message $\langle
id_{j}$, $x_{j}$, $w_{j_1}$, $\alpha_{j_1}\rangle$ and sends it to
$U_{i}$. As a response, $U_{i}$ computes $\langle id_{i}$,
$w_{i_2}$, $\alpha_{i_2}\rangle$ and sends it to $\mathcal{A}$. In
response, $\mathcal{A}$ computes $\langle id_{j}$, $w_{j_2}$,
$\alpha_{j_2}\rangle$ and sends it to $U_{i}$.  However,
$\mathcal{A}$ cannot compute the session key $SK = H_{3}(sid$,
$\sigma_{i_1}$, $\sigma_{i_2}$, $\sigma_{j_1}$, $\sigma_{j_2})$
because $\sigma_{i_1}$, $\sigma_{i_2}$, $\sigma_{j_1}$, and
$\sigma_{j_2}$ are unknown to him/her due to private key $(s_{j}$,
$e_{j})$ and secret random samples $(r_{i}$, $f_{i})$. Accordingly,
our LB-2PAKA protocol withstands the KCI attack.

\subsubsection{\textbf{Unknown Key-Share (UKS) Attack}}~Assume that
$U_{i}$ and $U_{j}$ generate a fresh and common session key $SK =
H_{3}(sid$, $\sigma_{i_1}$, $\sigma_{i_2}$, $\sigma_{j_1}$,
$\sigma_{j_2})$ among them in a session. In our LB-2PKA protocol,
neither entity believe that $SK$ is shared with an adversary
$\mathcal{A}$. Because, both of $U_{i}$ and $U_{j}$ compute the
session key $SK$ after mutually authenticating each other by
verifying whether
$\alpha_{j_1}^{\prime}\stackrel{?}{=}\alpha_{j_1}$,
$\alpha_{i_1}^{\prime}\stackrel{?}{=}\alpha_{i_1}$,
$\alpha_{j_2}^{\prime}\stackrel{?}{=}\alpha_{j_2}$ and
$\alpha_{i_2}^{\prime}\stackrel{?}{=}\alpha_{i_2}$ are correct.
Hence, our LB-2PKA protocol is free from the UKS attack.

\subsubsection{\textbf{Ephemeral Secret Leakage (ESL) Attack}}~
In our LB-2PKA protocol, $U_{i}$ and $U_{j}$ generate a fresh and
common session key $SK = H_{3}(sid$, $\sigma_{i_1}$, $\sigma_{i_2}$,
$\sigma_{j_1}$, $\sigma_{j_2})$ between then in each session, where
the computation of $\sigma_{i_1}$ and $\sigma_{j_1}$ depends on the
private keys $(s_{i}$, $e_{i})$ of $U_{i}$ and  $(s_{j}$, $e_{j})$
of $U_{j}$. Besides, computation of $\sigma_{i_1}$ and
$\sigma_{j_1}$ depends on the random samples $(r_{i}$, $f_{i})$
selected by $U_{i}$ and $(r_{j}$, $f_{j})$ selected by $U_{j}$.
Therefore, an adversary $\mathcal{A}$ is unable to generate the
session key $SK$ even if $(r_{i}$, $f_{i})$ and $(r_{j}$, $f_{j})$
are known to him/her. Thus, our LB-2PAKA protocol withstands the ESL
attack.

\subsubsection{\textbf{Perfect Forward Secrecy (PFS)}}~Assume that
the private keys $(s_{i}$, $e_{i})$ of $U_{i}$ and $(s_{j}$,
$e_{j})$ of $U_{j}$ are somehow disclosed to an adversary
$\mathcal{A}$. Despite having the private keys of $U_{i}$ and
$U_{j}$, $\mathcal{A}$ cannot compute the $SK = H_{3}(sid$,
$\sigma_{i_1}$, $\sigma_{i_2}$, $\sigma_{j_1}$, $\sigma_{j_2})$,
which is computed between $U_{i}$ and $U_{j}$ in a session. Because
the random samples $(r_{i}$, $f_{i})$ selected by $U_{i}$ and
$(r_{j}$, $f_{j})$ selected by $U_{j}$ are still unknown to
$\mathcal{A}$. Thus, our LB-2PKA protocol provides the PFS of the
session key.

\subsubsection{\textbf{No Key Control (NKC)}}~In our LB-2PKA protocol,
$U_{i}$ and $U_{j}$ establish a common session key $SK = H_{3}(sid$,
$\sigma_{i_1}$, $\sigma_{i_2}$, $\sigma_{j_1}$, $\sigma_{j_2})$ by
incorporating the the random samples $(r_{i}$, $f_{i})$ selected by
$U_{i}$ and $(r_{j}$, $f_{j})$ selected by $U_{j}$. Therefore, none
of $U_{i}$ and $U_{j}$ alone can determine the session key. Hence,
we can conclude that our LB-2PKA protocol provides NKC property of
the session key.

\section{Performance Analysis}
\label{L6}This section is devoted to present the performance
analysis of our LB-2PAKA protocol against the efficiency measurement
attributes (i) execution cost, and (ii) communication cost.
According to \cite{Saraswathy2017,FengHe2018}, we choose
$\mbox{log}_{2}\beta = 17.1$ for the Gaussian sampling distribution.
We use the SHA-512 hash function for hashing operation. The length
of the digest of the SHA-512 function is 512 bits. We choose the
length of $n$, $q$, and an element in $\mathbb{R}_{q}$ are
1024~bits, 12289~bits, 4096~bits, respectively. As discussed in
section \ref{L12} that Zhang et al.'s protocol \cite{ZhangJ2015} is
an unauthenticated two-party key agreement protocol; therefore, we
are not comparing it with our LB-2PAKA protocol. Accordingly, we
only estimate the execution cost and communication cost of our
LB-2PAKA protocol in this section.

\subsection{Execution Cost} \label{L61}
Here, we estimated the execution cost of our LB-2PAKA protocol. To
calculate the running time of our LB-2PAKA protocol, we consider
different cryptographic operations as follows: $T_{Ge}$, $T_{smul}$,
$T_{pmul}$, $T_{pma}$, $T_{cha}$ and $T_{H}$ denote the running time
time to draw a sample value from the $\chi_{\beta}$ distribution,
execute a component-wise multiplication with scalar operation in
$\mathbb{R}_{q}$, to execute a component-wise multiplication
operation in $\mathbb{R}_{q}$, to execute a component-wise
multiplication and addition operation in $\mathbb{R}_{q}$, to
execute a \textbf{Cha} operation in $\mathbb{R}_{q}$ and to
calculate a digest value for the SHA-512 hash function $H$,
respectively. In comparison, we ignored the
$\mbox{\textbf{Mod}}_{2}$ operation in $\mathbb{R}_{q}$ since it is
equivalent AND operation and thus its running time is ignorable.

The implementation of our LB-2PAKA protocol simulated using
LatticeCrypto Library \cite{LCL2019}. Based on the experimental
setup given in \cite{FengHe2018}, it is assumed that $U_{i}$ and
$U_{j}$ executes the proposed LB-2PAKA protocol using a Dell desktop
PC with the following configurations (i) Operating system: Windows
10 Professional, (ii) CPU: 3.4-GHz Intel(R) Core(TM) i7-6700
processor, and (iii) RAM: 8GB. The mode of message exchange between
$U_{i}$ and $U_{j}$ is Bluetooth communication. In this environment,
the average running time of different ideal lattice-based operations
(in nanoseconds), which have been estimated over 100000 executions,
are given in Table~\ref{tab2}.

%%%%%%%%%%%%%%%%%%%%%%%%%%%%%%%%%%%%%%%%%%%%%%%%%%%%%%%%%%%%%%%%%%%%%%%%%%%%%
%%%%%%%%%%%%%%%%%%%%%%%%%%%%%%%%%%%%%%%%%%%%%%%%%%%%%%%%%%%%%%%%%%%%%%%%%%%%%
\begin{table}[h]
\centering \caption{Execution cost} \label{tab2} \scalebox{1}{
\begin{tabular}{|l|c|c|c|c|c|c|}
\hline
\textbf{Notation}   &  $T_{Ge}$    & $T_{smul}$   & $T_{pmul}$   & $T_{pma}$    & $T_{cha}$    & $T_{H}$     \\
\hline
\textbf{Time}       & $73.503~ns$  & $00.298~ns$  & $00.307~ns$  & $02.549~ns$  & $00.698~ns$  & $14.09~ns$  \\
\hline
\end{tabular}}
\end{table}
%%%%%%%%%%%%%%%%%%%%%%%%%%%%%%%%%%%%%%%%%%%%%%%%%%%%%%%%%%%%%%%%%%%%%%%%%%%%%
%%%%%%%%%%%%%%%%%%%%%%%%%%%%%%%%%%%%%%%%%%%%%%%%%%%%%%%%%%%%%%%%%%%%%%%%%%%%%
In key agreement phase, the initiator $U_{i}$ draws two random
samples $r_{i}$, $f_{i}$ $\xleftarrow[]{R}\chi_{\beta}$ and
generates the message $\langle id_{i}$, $x_{i}$, $w_{i_1}$,
$\alpha_{i_1}\rangle$. The execution cost for the message $\langle
id_{i}$, $x_{i}$, $w_{i_1}$, $\alpha_{i_1}\rangle$ is $2T_{Ge} +
T_{smul} + T_{pma} + T_{pmul} + T_{cha} + T_{H}\approx164.948~ns$.
After receiving the message $\langle id_{i}$, $x_{i}$, $w_{i_1}$,
$\alpha_{i_1}\rangle$, the responder $U_{j}$ verifies it by checking
whether $\alpha_{i_1}^{\prime}\stackrel{?}{=}\alpha_{i_1}$ is
correct and the associated execution cost is $T_{H}\approx14.09~ns$.
In addition, the responder $U_{j}$ performed the similar
computations to generated the message $\langle id_{j}$, $x_{j}$,
$w_{j_1}$, $\alpha_{j_1}\rangle$ and hence the execution cost is
$164.948~ns$. After receiving the message $\langle id_{j}$, $x_{j}$,
$w_{j_1}$, $\alpha_{j_1}\rangle$, $U_{i}$ verifies it by checking
whether $\alpha_{j_1}^{\prime}\stackrel{?}{=}\alpha_{j_1}$ is
correct and the associated execution cost is $T_{H}\approx14.09~ns$.
In addition, $U_{i}$ computes the messages $\langle id_{i}$,
$w_{i_2}$, $\alpha_{i_2}\rangle$ and its execution cost is $T_{pma}
+ T_{cha} + T_{H}\approx17.337~ns$. After receiving the message
$\langle id_{i}$, $w_{i_2}$, $\alpha_{i_2}\rangle$, $U_{j}$ verifies
it by checking whether
$\alpha_{i_2}^{\prime}\stackrel{?}{=}\alpha_{2_1}$ is correct and
the associated execution cost is $T_{H}\approx14.09~ns$. Next,
$U_{j}$ computes the message $\langle id_{j}$, $w_{j_2}$,
$\alpha_{j_2}\rangle$ and the corresponding execution cost for this
message is $T_{pma} + T_{cha} + T_{H}\approx17.337~ns$. In addition,
$U_{j}$ computes a session key $SK = H_{3}(sid$, $\sigma_{i_1}$,
$\sigma_{i_2}$, $\sigma_{j_1}$, $\sigma_{j_2})$ and the execution
cost for it is $T_{H}\approx14.09~ns$. The execution cost for
$U_{i}$ to verify the message $\langle id_{j}$, $w_{j_2}$,
$\alpha_{j_2}\rangle$ by executing whether
$\alpha_{j_2}^{\prime}\stackrel{?}{=}\alpha_{j_2}$ is
$T_{H}\approx14.09~ns$. The execution cost for $U_{i}$ to compute
the session key $SK$ is $T_{H}\approx14.09~ns$. Therefore, the
overall execution cost of our LB-2PAKA protocol is $2\times(164.948
+ 3\times14.09 + 17.337)\approx224.555~ns$.

\subsection{Communication Cost} \label{L62}
To compute a session key, the initiator $U_{i}$ sends the messages
$\langle id_{i}$, $x_{i}$, $w_{i_1}$, $\alpha_{i_1}\rangle$ and
$\langle id_{i}$, $w_{i_2}$, $\alpha_{i_2}\rangle$ to the responder
$U_{j}$. Then the responder $U_{j}$, sends the messages $\langle
id_{j}$, $x_{j}$, $w_{j_1}$, $\alpha_{j_1}\rangle$ and $\langle
id_{j}$, $w_{j_2}$, $\alpha_{j_2}\rangle$ to the initiator $U_{i}$.
Here $\{x_{i}$, $x_{j}\}\in\{0,1\}^{4096}$, $\{w_{i_1}$, $w_{i_2}$,
$w_{j_1}$, $w_{j_2}\}\in\{0,1\}$, $\{\alpha_{i_1}$, $\alpha_{i_2}$,
$\alpha_{j_1}$, $\alpha_{j_2}\}\in\{0,1\}^{512}$. Therefore, the
communication overhead of our LB-2PAKA protocol is estimated as
$4096\times 2 + 1\times 4 + 512\times 6 = 10244$~bits per session.

\section{Conclusion}
\label{L7}We design an LB-2PAKA protocol with the hardness
assumption of the RLWE problem. In our protocol, two users can set
up a shared session key for each session among them over an insecure
network, which is monitored by an adversary. Our LB-2PAKA protocol
offers provable security in the random oracle model. Our provable
security analysis showed that the probability of an adversary to
infringement the semantic security of the session key is negligible,
and the time need is more than the time required to break the RLWE
problem. We also examine the execution time of our LB-2PAKA protocol
using the LatticeCrypto Library. Besides, we analyze the
communication overhead of our LB-2PAKA protocol. We believe that our
LB-2PAKA protocol will be more suitable than the traditional 2PAKA
protocols for many Internet-based applications in post-quantum
environments.

%\bibliographystyle{elsarticle-num}
\bibliographystyle{plain}
\bibliography{2paka}

\end{document}
